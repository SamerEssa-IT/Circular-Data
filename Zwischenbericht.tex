% !TeX program = xelatex
% Jan Agatz

\documentclass{scrartcl}

\usepackage[T1]{fontenc}
\usepackage[ngerman]{babel}
\usepackage{fontspec}
\usepackage{enumitem}
\usepackage{hyperref}

%Formatierung des Dokuments
\usepackage[right=3cm]{geometry}

%Kopfzeile
%Kann an andere Handouts angepasst werden
\usepackage{fancyhdr}
\pagestyle{fancy}
\lhead{R-Projekt \glqq Circular Data\grqq \\
		\small Betreuer: Sergio Brenner Miguel} %Betreuer ergaenzt
\rhead{\small \today \\
		Agatz, Goldenbaum, Schneider, Sheich Essa} %Datum hinzugefuegt

%--------------------------------------------------------------

\begin{document}
\section*{Zwischenbericht}
\noindent\textbf{Was wurde bisher gemacht?}
\begin{itemize}
	\item Das entsprechende \href{https://gitlab.com/Jan-Agatz/circulardata}{GitLab-Repository} erstellt
	\item Plot-Funktionen implementiert
	\begin{itemize}
			\item Circular plot
			\item Circular Histogram
			\item Rose diagram
		  \end{itemize}
	\item Simulationsfunktion implementiert
	\item Dichten implementiert
	\begin{itemize}
		\item Gleichverteilung auf dem Kreis
		\item Von-Mises-Verteilung
	\end{itemize}
	\item Nicht-parametrische Regression implementiert
\end{itemize}

%--------------------------------------------------------------
%--------------------------------------------------------------
%--------------------------------------------------------------

\noindent\textbf{Wie ist der Zustand aktuell?} \\
\noindent Aktuell werden die Funtionen des Pakets implementiert. Der Fokus liegt auf harmonisierten In- und Outputs. \\

%--------------------------------------------------------------
%--------------------------------------------------------------
%--------------------------------------------------------------

\noindent\textbf{Was muss noch gemacht werden?}
\begin{itemize}
	\item Restliche Funktionen einfügen
	\begin{itemize}
		\item Weitere Dichten
		\item Kerndichteschätzer
		\item Grafen in Plotfunktionen implementieren
	\end{itemize}
	
	\item Formalia einrichten
	\begin{itemize}
		\item Beschreibung des Projektes in DESCRIPTION.txt
		\item Richtige Lizenz verwenden
		\item Richtige Namenskonventionen verwenden
		\item Funktionen in korrekten Dateien speichern
	\end{itemize}

	\item Harmonisierung der Funktionsargumente und -namen, sowie der verwendeten Datentypen
	\item Tests
	\item Dokumentation
	\item Shiny-App
	\item Vignetten
\end{itemize}

%--------------------------------------------------------------
%--------------------------------------------------------------
%--------------------------------------------------------------

\noindent\textbf{Zeitplan}
\begin{itemize}
	\item[-28.07.] Funktionalitäten implementieren (plots, simulations, density estimators) + grober Aufbau des Pakets
	\item[29.07.-02.08.] Fertigstellung des Pakets (Tests, Vignetten, Dokumentation,...)
	\item[03.08.-05.08.] Shiny-App
	\item[06.08.-09.08.] Video erstellen + Projekt abgeben \textbf{(Deadline: 11.08.)}
	\item[10.08.-12.08.] Bericht schreiben \textbf{(Deadline: 15.08.)}
\end{itemize}
     
\end{document}

